%
% $Id: buttonServer.tex,v 1.1 2002/09/14 15:44:40 rstone Exp $
%
\NeedsTeXFormat{LaTeX2e}

\documentclass{article}

\addtolength{\parskip}{1ex plus0.5ex minus0.2ex}

\addtolength{\textwidth}{2cm}

\begin{document}

\section{The button server}

Version of this document is
\begin{verbatim}
$Id: buttonServer.tex,v 1.1 2002/09/14 15:44:40 rstone Exp $
\end{verbatim}
 
\subsection{How the button server works} 
 
A program dealing with the button server typically has four steps (take a look at
\begin{center} 
  \texttt{\~{}bee/src/buttons/example.c}
\end{center}
for an example):

\begin{enumerate}
  
\item \texttt{buttonRegister( );} 
  
  Registers all new commands that we need for dealing with the button server.
  
\item \texttt{buttonConnect( 1 );} 
  
  Waits until we have connected to the button server.
  
\item \texttt{registerButtonStatusCallback( myButtonStatusCallback );} 
  
  Registers the function \texttt{myButtonStatusCallback}. The button server
  calls this function once a button on the robot changes it's state.

\item \texttt{buttonSubscribeStatus( 1 );}
  
  We subscribe to every status update of the button server. Every time a
  button on the robot changes it's state, the button server calls our function
  \texttt{myButtonStatusCallback}.

\end{enumerate}
 
If you want to use the button server, follow the above four steps, include
\texttt{buttonClient.h} in your source and link with \texttt{-lbuttonClient}.

\subsection{Commands for the button server}

\begin{description}
  
\item \texttt{void buttonRegister( );}

  Registers the button server.
 
\item \texttt{int buttonConnect( int wait\_till\_established );} 
  
  Connects to the button server. If \texttt{wait\_till\_established} is
  \texttt{1}, we wait until connection with the button server has been
  established. Otherwise we proceed.
 
\item \texttt{void registerButtonStatusCallback( buttonStatusCallbackType fcn
    );} 
  
  Registers a function \texttt{fcn}. The button server calls this function
  every time a button on the robot changes it's state.

\item \texttt{void buttonSetButtons( int red\_light\_status,\\
    int yellow\_light\_status,\\
    int green\_light\_status,\\
    int blue\_light\_status,\\
    int left\_kill\_switch\_light\_status,\\
    int right\_kill\_switch\_light\_status );}
  
  Changes the status of multiple buttons. Status for each button is one of\\
  \texttt{BUTTON\_LIGHT\_STATUS\_OFF}, \texttt{BUTTON\_LIGHT\_STATUS\_ON},
  \texttt{BUTTON\_LIGHT\_STATUS\_FLASHING},\\
  \texttt{BUTTON\_LIGHT\_STATUS\_FLASHING\_TILL\_PRESSED},
  \texttt{BUTTON\_LIGHT\_STATUS\_ON\_TILL\_PRESSED},
  \texttt{BUTTON\_LIGHT\_STATUS\_OFF\_TILL\_PRESSED},
  \texttt{BUTTON\_LIGHT\_STATUS\_TOGGLE\_ON},\\
  \texttt{BUTTON\_LIGHT\_STATUS\_TOGGLE\_OFF} or
  \texttt{BUTTON\_LIGHT\_STATUS\_DONT\_CHANGE}.

\item \texttt{void buttonSetButton( int button, int status );}
  
  Sets the status of button \texttt{button} to status \texttt{status}.
  \texttt{button} is one of \texttt{BUTTON\_LEFT\_KILL},
  \texttt{BUTTON\_RIGHT\_KILL}, \texttt{BUTTON\_RED}, \texttt{BUTTON\_YELLOW},
  \texttt{BUTTON\_GREEN} or \texttt{BUTTON\_BLUE}. For \texttt{status} see
  \texttt{buttonSetButtons(\dots)}.

\item \texttt{void buttonStartCuteThing( );}
  
  Flashes the red, yellow, green and blue button.

\item \texttt{void buttonRequestStatus( );}
  
  Requests a button status. The button server calls the function \texttt{fcn}
  that was previously registered with
  \texttt{registerButtonStatusCallback(\dots)}.

\item \texttt{void buttonSubscribeStatus( int number );}
  
  If \texttt{number} is not zero, subscribe to every \texttt{number}-th status
  update from the button server. Every time a button on the robot changes it's
  state, the button server calls the function \texttt{fcn} that was previously
  registered with \texttt{registerButtonStatusCallback(\dots)}.

\end{description}

\subsection{Parameters for the button server}

\begin{description}

\item \texttt{-test=0}
  
  Do not test the buttons. This is the default.

\item \texttt{-test=1}
  
  Test the buttons. Flash the red, yellow, green and blue button. Does not use
  TCX.

\item \texttt{-simulator=0}
  
  Process TCX commands immediately. This is the default.

\item \texttt{-simulator=1}

  Wait for a TCX command. Probably.

\end{description}
 
\subsection{The example program \texttt{buttonExample}}
 
Start \texttt{tcxServer} and \texttt{buttonServer} (the button server of
course on the machine that has control over the buttons). Now start the program
\texttt{buttonExample}.

\texttt{buttonExample} will connect to the button server, register a status
callback function (in which you can see how the buttons changed their state)
and subscribe to the status update. It then flashes the red, yellow, green and
blue button with \texttt{buttonStartCuteThing( )}.

\end{document}