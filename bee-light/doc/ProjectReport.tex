% This file was converted from HTML to LaTeX with Nathan Torkington's
% html2latex program
% Version 0.9c
\documentstyle{article}
\begin{document}
{\bf \par Re-implementation of BeeSoft Light}A distributed
architecture for mobile robot navigation\par \par Jianjung Ying\par
\par Information Networking Institute\par \par Carnegie Mellon
University\par \par \par \par Introduction\par {\it \par
  Re-implementation of BeeSoft Light} is one of the student projects
of course 15-822, System Design \& Implementation Practicum, offered
by Computer Science Department, at Carnegie Mellon University. The
goal of this course is to provide a hands-on learning opportunity to
acquire skills relevant to the design, implementation, evaluation and
tuning of complex systems. This course is project-based, with each
student doing a different project. The projects are are
\&quot;real\&quot; rather than contrived. Every project addresses the
needs of some \&quot;customers\&quot;, which may be CS or Robotics
research projects. Sebastian Thrun, assistant professor of Computer
Science Department at CMU, sponsors {\it \&quot;Re-implementation of
  BeeSoft Light\&quot;} project. This project is to re-design and
re-implement BeeSoft Light API library. Then his Robot Learning
Laboratory can write programs to navigate their robots more easily
than before.\par \par Architectural Overview \par \par The overall
software architecture consists of several modules (processes) running
in parallel on on-board laptops and off-board workstations. The
software modules communicate using {\it TCX}, a centralized
communication protocol for point-to point socket communication. The
overall software architecture is showed below:\par \par \par \par This
architecture of distributed module provides a robust solution to robot
navigation. Time-critical software (e.g. all device drivers), and
software that is important for the safety of the robot (e.g. collision
avoidance in colliServer), are running on the robot's on- board
laptops. Higher-level software, such as the Planner and LOCALIZE is
running on the off-board workstations. Even the communication links
are not reliable and various events can temporarily delay the message
flow or reduce the available computational time, the robot can always
have robust behavior. Each module provides certain functionality, but
not all modules are required to run:\par \begin{itemize}{\bf \item
    tcxServer} (not shown in the diagram) coordinates communications
  among all other modules running on all computers in the robot
  system. TCX is a decentralized communication protocol for
  point-to-point socket communication. Processes invoke TCX API to
  communicate with each other.\item {\bf pioneer/baseServer} issues
  commands to the robot's wheeled base. It can move forwards and
  backwards, as well as rotational motion - clockwise and
  counterclockwise - around the robot's center axis.\item {\bf
    laserServer} administrates the equipped laser device. \item {\bf
    cameraServer }administrates the equipped camera device.\item {\bf
    colliServer} sends commands to pioneer/baseServer in guarded
  motion mode, that is, incorporating built-in collision avoidance.
  You give the module a target point, either a relative target point
  or an absolute target point. The robot will try its best to avoid
  obstacles it can perceive. It uses sonar and laser devices to know
  the robot's environment. It knows how far the obstacles are by
  emitting sonar and laser and then collecting the reflection. Other
  modules could also subscribe to those sensor values. Notice that
  {\it pioneer/baseServer} does not include automatic obstacle
  avoidance. They just order the robot to move.\item {\bf Planner} is
  the robot motion management and exploration software. It helps the
  robot plan globally optimal motion and optimal paths. When Planner
  receives a target, it will compute the optimal path and then give
  several intermediate goal points to the colliServer. Planner is
  running on an off-board workstation. While colliServer provides
  local collision avoidance, Planner provides global path finding. To
  do so, it needs a map of the robot's environment.  We could use a
  program called Mapper to establish maps, but it is beyond the scope
  of this project.\item {\bf LOCALIZE }is a module which continuously
  estimates the robot's position in x-y-q space, where x and y are the
  coordinates of the robot in a 2D Cartesian coordinate system and q
  is its orientation. colliServer has a base coordinate system which
  is different from map coordinate system. Moreover, robot may drift
  so that it will deviate from its target direction. LOCALIZE monitors
  the robot and keeps{\bf }computing corrected{\bf }coordinate
  parameters. It uses sensor readings from the robot's sensors and
  actions commands to keep estimating positions. Then task planning
  modules, such as user modules, can use those parameters to transform
  map coordinates to robot's base coordinates, which are colliServer's
  acceptable command arguments.  \item {\bf SET\_ROBOT }gives the
  robot an initial point in map coordinate system. Otherwise, LOCALIZE
  does not know where to locate the robot initially.
\end{itemize}\par TCX grants the BeeSoft system scalability of adding
more modules. BeeSoft package includes some other modules to
administrate onboard devices, such as pantiltServer, armServer,
speechServer. But not all of the robots have those devices. A
programmer can write USER module to ask the robot to perform a task.
This module also uses TCX mechanism to communicate with other modules.
In this architecture, there could be many user modules running
concurrently. Modules request for services by passing messages to the
server modules and receive the reply messages. Each BeeSoft module
defines a number of messages. Other modules could ask this module for
services by sending messages through TCX API. BeeSoft API library
encapsulates the details of the underlying communication
infrastructure. You could write programs easily even if you do not
know TCX.\par \par Goal of this project\par \par The former section
gives an overall concept of BeeSoft Light. A user module programmer
needs to have an understanding of TCX communication mechanism. He
needs to create a thread and write sensor callback functions to
collect messages sent from other modules. He also needs to know the
module messages to order the robot to perform desired actions.\par
\par For a novice of BeeSoft Light, it takes long time to master TCX
mechanism and those module messages. Even for a professional, it is
time-consuming to write those required but always-the-same lines, such
as message handlers and TCX initialization. \par \par This project is
to re-design and re-implement BeeSoft Light API library to encapsulate
TCX communication and message passing architecture. Then it offers
robotics programmers ease of use, simplicity of architecture, and high
degree of transparency. \par \par \&nbsp;\par \par BeeSoft Light API
library functions\par \par The new BeeSoft Light API library consists
of those functions:\par {\bf \par BeeSoft Application
  Initialization:\par \begin{enumerate}\item beeInitialize() -
  initializes a user module process. \item  beeCameraConnect() -
  connects to cameraServer.\end{enumerate}{\bf \par Motion
  functions: }functions that control the robot's motion\par
\begin{enumerate}\item beeTranslateBy() - robot moves forward or
  backward by a distance\item  beeRotateBy()- robot rotates
  clockwise or counterclockwise by a degree\item 
  beeTranslatePositive() - robot keeps moving forward \item 
  beeTranslateNegative() - robot keeps moving backward\item 
  beeRotatePositive() - robot keeps rotating clockwise\item 
  beeRotateNegative() - robot keeps rotating counterclockwise 
\item beeStop() - robot stops\item  beeSetTransVelocity()- robot
  sets translation velocity \item  beeSetRotVelocity() - robot
  sets rotation velocity\item  beeSetTransAcceleration() - robot
  sets translation acceleration\item  beeSetRotAcceleration() -
  robot sets rotation acceleration\item  beeGetCurrentPosition()
  - returns the current position of the robot\item 
  beeGetCurrentSpeed() - returns the current speed of the robot
\item beeApproachRelative() - robot tries to move to a position
  without optimal path finding \item  beeApproachAbsolute() -
  robot tries to move to an absolute position without optimal path
  finding\item  beeGotoRelative () - robot moves to a relative
  position with optimal path finding\item  beeGotoAbsolute() -
  robot moves to an absolute position with optimal path finding
\end{enumerate}{\bf \par Sensor functions: }functions that retrieve
the sensor readings\par \begin{enumerate}\item
  beeSonarStartRegularUpdate() - starts the subscription of sonar
  readings from the robot\item  beeSonarStopRegularUpdate() -
  stops the subscription of sonar readings\item 
  beeGetSonarValue() - retrieves an array of sonar readings\item 
  The same groups of functions for other sensors: laser, infrared
\end{enumerate}{\bf \par Camera Library: }functions that retrieve
camera images\par \begin{enumerate}\item beeCameraSaveFile() -
  requests the cameraServer to capture and save images in local
  disk.\item  beeCameraLoadFile(){\bf - }requests the
  cameraServer to load images from a file.\item 
  beeCameraRequestShmId() - requests the cameraServer to capture and
  put an image to a shared memory.\item  beeCameraRequestImage()
  - requests the cameraServer to capture and transfer an image to the
  user module.\item  beeCameraSubscribe() - subscribes to images
  captured from the cameraServer.\item 
  beeCameraUnsubscribe(){\bf - }unsubscribes frames from a
  cameraServer. \end{enumerate}{\bf \par Callback functions:
  }functions register callback functions\par \begin{enumerate}\item
  beeSonarRegisterCallback() - registers a callback function to
  process the sonar reading every time it is read\item 
  beeIRRegisterCallback(){\bf - } registers a callback function to
  process the infrared reading every time it is read. \item 
  beeLaserRegisterCallback(){\bf - }registers a callback function to
  process the laser reading every time it is read.\item 
  beeCameraRegisterCallback - registers a callback function to process
  the camera image every time it is captured\end{enumerate}\par
For a detailed of these functions, please refer to the user manual.
\par \par Schedule of the project\par \begin{itemize}{\bf \item Check
    point 1 goal : }Get familiar with the underlying message transfer
  infrastructure (TCX) and design the API library functions.
\end{itemize}\begin{itemize}{\bf \item Check point 2 goal :} Finish
  the implementation of the core functions, write Unix manual pages
  for the core functions. \end{itemize}\begin{itemize}{\bf \item
    Check point 3 goal :} Finish the whole implementation and prepare
  the demo programs which could run on robots.\end{itemize}\par
Experience learned from the project\par \par The goal of this course
is to provide a hands-on learning opportunity to acquire skills
relevant to the design, implementation, evaluation and tuning of
complex systems. From the project, I learned the following
experience:\par \begin{itemize}{\bf \item More practice in
    programming}:\end{itemize}\par This project used g++
compiler in Linux machines. I used g++ because I need to overload
several functions of the same name. Each of them has different
parameter list. But implementation of those functions was
procedure-oriented, rather than object-oriented. I did not group those
functions into object methods because it is easy to think of the robot
as an object and those functions as robot's methods. Programmers who
are not familiar with object-oriented programming may code with the
BeeSoft Light API library easily. \par \par In this project, I also
had practice in thread creation and synchronization. Every module
needs a background thread to keep polling the TCX mechanism to
retrieve new-arrived TCX messages. It is not difficult to create a
thread and run it. But it will be a problem that the foreground thread
is retrieving the array of sensor readings while the background thread
is updating it. I need to use some semaphore mechanisms to prevent
data inconsistency.\par \par Several functions allow programmers to
register new callback functions. Locations of those callback functions
are stored and invoked every time the counterpart events are
triggered. This is an advanced programming skill and I took one hour
to find out a solution for C.\par \par I also learned more syntax of
Makefile. I knew how to create a library file. \par
\begin{itemize}{\bf \item System architecture design}
  \end{itemize}\par This semester I took Architecture for Software
Systems in Software Engineering Institute. I learned that a good
architecture design could contribute a good solution to the problem. I
learned an architecture model called Implicit-invocation and case
study of mobile robot. The characteristics of implicit-invocation are
its scalability and easy modification. It allows programmers to add
more modules without modifying existing modules much. BeeSoft Light
project is a good example of implicit-invocation model. The tcxServer
is responsible to dispatch messages to modules that registered to
listen to.\par \begin{itemize}{\bf \item Documentation}
  \end{itemize}\par I did not have a good habit in documenting my
code before this project. In my previous programming projects I
documented only for those in which documentation was graded. In the
first checkpoint I was recommended to write Unix manual pages. I
consulted some Linux experts and knew the syntax of Unix manual pages.
Because they structure what should be told to library users, they
enable me to plan and write my documents easily.\par
\begin{itemize}{\bf \item Hands-on experience in mobile robots}
  \end{itemize}\par This is my first time to run mobile robots and
program for them. There are many onboard devices on a robot, such as
laser sensors, sonar sensors, camera device, radio transmitter and
receiver. To run a robot, I need to take care of those devices. I need
to remember to recharge the power of the robot. \par \par When
designing the API library, I had to consider which primitive commands
should be included in the library. I need to stand in the perspective
of novice users in robotics programming. I finished the library design
by the first checkpoint. But after that I made some additions and
modifications. Through the process I got the experience to design a
library of functions. The clues I learned are (1) naming the functions
well to explain their functionality; (2) reducing the number of the
arguments; (3) documenting the functions for easy reference. \par \par
\par \par Troubles I had\par \begin{itemize}\item Workload from other
  courses affects my project schedule. \item  I relied too much
  on the Simulator. The Simulator does not simulate real interaction
  between the robot and its environment. So it may cause some false
  interactions between the environment and my test program. I had an
  experience in wasting an amount of time in finding a non-existing
  bug. \item  Bugs: Two bugs were with me for more than one
  month. The first bug was that when the user module terminated,
  tcxServer and other modules terminated too. The second bug was that
  beeApproachRelative() did not perform as it was expected to be. I
  guessed those bugs resulted from some server modules. Finally in the
  last week of the third checkpoint I found out where the problems
  were. The first bug resulted from an uninitialized variable. The
  second bug resulted from misunderstanding the beeApproachRelative()
  function. \end{itemize}\par Future work\par \par This project
includes library for robot navigation and camera. Some robots are
equipped with more devices, such as tactile sensors (detecting bumps),
speech device, pan/tilt head, and so forth. Another project could
re-design and re-implement library functions for those device. There
are also counterpart server modules managing those resources. User
modules could also use TCX messages to communicate with those server
modules. Client library functions hide the communication mechanisms
and provide programmers higher-level view of robotics programming.\par
\par Robot navigation has an important application in unattended
automobile. This project could have similar and interesting
application in \&quot;attended robot\&quot;. Since most of the robots
are equipped with a camera, programmers can use camera functions in
BeeSoft library to code their own \&quot;attended robot\&quot;
project. In this project, users control the robot with a joystick.
They can see what the camera is focusing on and change the robot's
motion speed and direction with the joystick. While a user push the
joystick, the program executes a beeSetTransVelocity() function
followed by a beeTranslatePositive() function. While a user pull the
joystick, the program executes a beeSetTransVelocity() function
followed by a beeTranslateNegative() function. Dragging the joystick
left causes the counterclockwise rotation of the robot, while dragging
the joystick right causes the clockwise rotation of the robot. \par
\par This application is very attractive for entertainment purpose. If
so, the baud rate for radio transmission needs to be taken care of.
The workstation communicates with the robot through radio connection.
Image streams will consume a large bandwidth and affect the quality of
the video. Moreover, each robot is equipped with a battery. It needs
to recharge frequently during a continuous use. The power consumed by
the camera needs to be considered too.\par \par What could have I done
to improve this project\par \begin{itemize}{\bf \item Got an early
    tutorial for this project}. I got a BeeSoft user manual in the
  first appointment with my sponsor. I also got a package of module
  programs to run a robot. There was a \&quot;missing link\&quot;
  between them so that I did not know how to start my project. I was
  taught how to run a robot in Simulator and some example programs in
  the second week. From then on I had stead progress. I think an early
  tutorial was important to me. And I hope the manual I developed
  could help future BeeSoft novices.\item {\bf Ran real robots early.}
  I learned how to run real robots one week before my third
  checkpoint. Before that I used Simulator to test my library. I need
  to take care of many things in real robot, such as power, switches
  of onboard devices. I made a stupid mistake in the poster session.
  The robot could not run because I did not turn on a radio switch. I
  lacked the overall knowledge of the power system of the real robot
  because there is no instruction manual for those details. If I had
  more practice in running real robots, I might have met those
  problems early and have gotten the solution.\item {\bf Set
    sub-checkpoints between main checkpoints}. I did most of the work
  when a checkpoint approached. The characteristic of the course is
  flexibility in schedule - one meeting per month. Sub-checkpoints
  might help the project progress better.\end{itemize}
  \end{document}

