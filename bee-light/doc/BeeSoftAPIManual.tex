% This file was converted from HTML to LaTeX with Nathan Torkington's
% html2latex program
% Version 0.9c
\documentstyle{article}
\begin{document}
{\bf \par BeeSoft User's Guide and Software Reference\par }\par
Jianjung Ying\par \par Robot Learning Laboratory\par \par Carnegie
Mellon University\par \par \par \par Preface\par \par BeeSoft is the
software that lets you navigate your robot. Each robot contains one or
more onboard computers that control its resources, such as its mobile
base, sonar, lasers, camera, pan/tilt head, and enable robot to move
forward along a specific heading, detecting a nearby obstacle, and
capture images from its camera.\par \par The overall software
architecture consists of several modules (processes) running in
parallel on on-board laptops and off-board workstations. Those server
modules manage robot resources well, empowering you to control those
resources as if they were local to your own program. The software
modules communicate using {\it TCX}, a centralized communication
protocol for point-to point socket communication. Your own user
application program communicates with the server modules, and thus
directs the robot's activities, via BeeSoft API (Application
Programming Interface) function calls. For example, when you want to
tell the robot to move forward a certain distance, you need only call
the appropriate function in the client API library. The client API
library function then determines which server modules should work
together and then instructs base server to initiate forward
motion.\par \par BeeSoft Architecture: the Server Modules\par
\begin{itemize}{\bf \item tcxServer} coordinates communications among
  all other modules running on all computers in the robot system. TCX
  is a decentralized communication protocol for point-to-point socket
  communication. Processes invoke TCX API to communicate with each
  other.\item {\bf pioneer/baseServer} issues commands to the robot's
  wheeled base. It can move forwards and backwards, as well as
  rotational motion - clockwise and counterclockwise - around the
  robot's center axis. You could run {\it pioneer} for little robots
  Marion and Robin, or {\it baseServer} for Amelia and Xavier. \item
  {\bf laserServer} administrates the equipped laser device. \item
  {\bf cameraServer }administrates the equipped camera device.\item
  {\bf colliServer} sends commands to {\it pioneer/baseServer} in
  guarded motion mode, that is, incorporating built-in collision
  avoidance. You give the module a target point, either a relative
  target point or an absolute target point. The robot will try its
  best to avoid obstacles it can perceive. It uses sonar and laser
  devices to know the robot's environment. It knows how far the
  obstacles are by emitting sonar and laser and then collecting the
  reflection. Other modules could also subscribe to those sensor
  readings. Notice that {\it pioneer/baseServer} do not include
  automatic obstacle avoidance. They just order the robot to
  move.\item {\bf Planner} is the robot motion management and
  exploration software. It helps the robot plan optimal motion and
  optimal paths. When Planner receives a target, it will compute the
  best path and then give several intermediate goal points to the
  colliServer. Planner is not necessary to run on an on-board laptop.
  While colliServer provides local collision avoidance, Planner
  provides global path finding. To do so, it needs a map of the
  robot's environment.  We could use a program called Mapper to
  establish maps, but it is beyond the scope of this document.\item
  {\bf LOCALIZE }is a module which continuously estimates the robot's
  position in x-y-q space, where x and y are the coordinates of the
  robot in a 2D Cartesian coordinate system and q is its orientation.
  colliServer has a base coordinate system which is different from map
  coordinate system. Moreover, robot may drift so that it will deviate
  from its target direction. LOCALIZE monitors the robot and keeps{\bf
    }computing corrected{\bf }coordinate parameters. It uses sensor
  readings and action commands to keep estimating positions. Then task
  planning modules, such as user modules, can use those parameters to
  transform map coordinates to robot's base coordinates, which are
  accepted by colliServer.  \item {\bf SET\_ROBOT }gives the robot an
  initial point in map coordinate system. Otherwise, LOCALIZE does not
  know where to locate the robot initially.\end{itemize}\par TCX
grants the BeeSoft system scalability of adding more modules. BeeSoft
package includes some other modules to administrate onboard devices,
such as pantiltServer, armServer, speechServer. But not all of the
robots have those devices. A programmer can write a USER module to ask
the robot to perform a task. This module also uses TCX mechanism to
communicate with other modules. In this architecture, there could be
many user modules running concurrently. Modules request for services
by passing messages to the server modules and receive the reply
messages. Each BeeSoft module defines a number of messages. Other
modules could ask this module for services by sending messages through
TCX API. BeeSoft API library encapsulates the details of the
underlying communication infrastructure. You could write programs
easily even if you do not know TCX.\par \par Running robot\par \par
The following steps guide you to run your robot. In Robot Learning
Laboratory, I assume that you are using the little robots, Marion or
Robin. \par \begin{enumerate}\item Switch on main power, radio power,
  laser power, and motor power. Please contact members in the lab. to
  learn the location of those switches.  \item Run the following
  programs in the robot's onboard laptops. Each program needs a shell
  to run. You could use \&quot;ssh marion\&quot; or \&quot;ssh
  robin\&quot; to login to the computers. Then remember to type
  \&quot;setenv TCXHOST marion\&quot; or \&quot;setenv TCXHOST
  robin\&quot; so that the server modules can locate the tcxServer.
  \end{enumerate}\par Change to the beeSoft \&quot;bin\&quot;
directory first. Then run:\par \begin{itemize}\item tcxServer 
\item pioneer 
\item laserServer -pioneer 
\item colliServer +laserServer
  -pioneer \end{itemize}\par After those server modules run, you
could ask the robot to move locally without a map. Since you have not
implemented any user module program yet, you could use colliServer
commands to order the robot.\par \par In the colliServer shell, type
\&quot;CTR 100 0\&quot;. You ask the robot to move one meter (100
centimeters ahead).\par \par CTR stands for \&quot;control translation
and rotation\&quot;. The first argument is the distance, in
centimeters, you want the robot to move along its current heading: the
heading towards which it is already pointed. A positive value in this
argument sends the robot forward. The second argument is the distance,
again in centimeters, that you want the robot to move perpendicular to
its heading.\par \par Note that colliServer will attempt to move the
robot to the point one meter directly ahead of itself, that is, one
meter in the direction the robot is facing when you issue the CRT
command. If that spot is blocked (occupied by an obstacle), the robot
will continue maneuvering to try to reach that point.\par \par To ask
your robot to go back, type \&quot;CTR -100 0\&quot;. \par \par Type
\&quot;?\&quot; in colliServer shell to learn more commands.\par \par
BeeSoft robot has a capability of optimal path finding. To do so, you
need a map of the environment. You could use Mapper to establish the
map, but it is beyond the scope of this document.\par \par To enable
the optimal path finding functionality, run the following server
modules as well. But they are not required to run in the robot's
onboard laptop. You could run them in a remote workstation. You need
an X-window workstation to display windows those processes create.
Change to the \&quot;localize\&quot; directory first,\par \par \#cd
(BeeSoft directory)/data/localize/localize\par \begin{itemize}\item
  ../../../bin/LOCALIZE laser.ini  \item ../../../bin/plan 
\item ../../../bin/DISPLAY\_LOCALIZE -gmap xxxx.map 
\end{itemize}\par You should specify a map file in laser.ini and when
you run DISPLAY\_LOCALIZE.\par \par To use camera service, you also
need to run\par \begin{itemize}\item cameraServer  
\end{itemize}\par Note that it needs a root shell to run
cameraServer.\par \par Running on Simulator\par \par Running your
robot is very interesting and exciting. Who do not like to watch the
cute robot moving and dancing around? However, you might not want to
always run your robotics programs while they are still in development
and debugging stage. Simulator is a good program that allows you to
test your program without running a real robot.\par \par Try the
following procedure in an X-window workstation.\par
\begin{itemize}\item tcxServer\item
  simulator -map floor.sim\item  colliServer -noindex +simulator +laser\item LOCALIZE laser.ini\item plan\item SET\_ROBOT -gmap floor.map 
\end{itemize}\par floor.sim should be in
(YOUR\_BEE\_DIRECTORY)/data/localize/localize\par \par Robot
programming\par \par There are several demo programs in
BeeSoftAPI/example directory. Compile and run those programs as your
first step in robot programming.\par \par Every program starts with
beeInitialize() function. This function initializes connection with
TCX server and registers message handlers for your user module
application. Your BeeSoft program must also start with this
function.\par \par The prototypes of those functions are defined in
{\it bee.h}. Include the file in the beginning of your C program.\par
\par To make your program, add -lpthread -lbee -ltcx into the library
list.\par \par BeeSoft API library\par \par This section lists all of
the BeeSoft API library functions with descriptions. Most of those
functions return a \&quot;bool\&quot; (boolean) value. TRUE value
means that this function has sent a message to a server module. It
does not promise that the robot will execute the order. Those
functions return FALSE when some required server modules are not
running or connected. \par {\bf \par BeeSoft Application
  Initialization:\par \begin{enumerate}\item void beeInitialize(char*
    moduleName = \&quot;USER\&quot;) - initializes a BeeSoft user
  process. The \&quot;{\it moduleName\&quot;} argument specifies the
  name of your user process module. It will initialize tcx
  communication, register call back functions, connect to TCX server
  and colliServer, and try to connect to LOCALIZE and Planner. If you
  do not provide the argument, the default module name is "USER". You
  should give names if you have more than one user process running.
  This function returns no value.\item {\bf bool
    beeCameraConnect(void)} - connects to cameraServer. 
\end{enumerate}{\bf \par Motion functions : }functions that control
the robot's motion\par \begin{enumerate}{\bf \item bool
    beeTranslateBy(float distance)} - moves a BeeSoft robot along a
  straight line The line has the same direction with the robot's local
  orientation. The {\it \&quot;distance"} argument specifies the
  distance to move. The measure of distance is centimeter. Positive
  value means forward movement, while negative value means backward
  movement. The robot will then stop in expected point or less if it
  meets an obstacle. This function does not make collision avoidance.
  The robot can not move around an obstacle. It just moves a distance
  ahead or backward. \item {\bf bool beeRotateBy(float degree)} -
  rotates a BeeSoft robot's orientation The {\it \&quot;degree"}
  argument specifies the degree to rotate. The measure is degree.
  Positive value means clockwise rotation, while negative value means
  counterclockwise rotation.\item {\bf bool beeTranslatePositive()} -
  moves a BeeSoft robot along a straight line.  The line has the same
  direction with the robot's local orientation. Program can use
  "beeSetTransVelocity()" and "beeSetTransAcceleration()" to control
  the robot. The robot keeps moving ahead with the velocity and
  acceleration. The robot will not move around an obstacle. It will
  stop.\item {\bf bool beeTranslateNegative()} - moves a BeeSoft robot
  along a straight line.  The line has the same direction with the
  robot's local orientation. Program can use "beeSetTransVelocity()"
  and "beeSetTransAcceleration()" to control the robot. The robot
  keeps moving backward with the velocity and acceleration. The robot
  will not move around an obstacle. It will stop.\item {\bf bool
    beeRotatePositive()} - rotates a BeeSoft robot's orientation
  clockwise. Program can use "beeSetRotVelocity()" and
  "beeSetRotAcceleration()" to control the robot. The robot keeps
  rotating with the velocity and acceleration until next "beeStop()"
  .\item {\bf bool beeRotateNegative()} - rotates a BeeSoft robot's
  orientation counterclockwise. Program can use "beeSetRotVelocity()"
  and "beeSetRotAcceleration()" to control the robot. The robot keeps
  rotating with the velocity and acceleration until next "beeStop()"
  .\item {\bf bool beeStop()} - stops a robot when it is moving or
  rotating.\item {\bf bool beeSetTransVelocity(float velocity)} - sets
  the translation velocity of a BeeSoft robot.  The {\it "velocity"}
  argument specifies the velocity to set. The measure of velocity is
  centimeter per second. Positive value means forward velocity, while
  negative value means backward velocity. Programmer can then use
  "beeTranslatePositive()" or "beeTranslateNegative()" to move the
  robot. It will keep moving with the velocity. The velocity will
  decrease to zero when it meets an obstacle and stops.\item {\bf bool
    beeSetRotVelocity(float velocity)} - sets the rotation velocity of
  a BeeSoft Light robot.  The {\it "velocity"} argument specifies the
  velocity to set. The measure of velocity is degree per second.
  Positive value means clockwise velocity, while negative value means
  counterclockwise velocity. Programmer can then use
  "beeRotatePositive()" or "beeRotateNegative()" to rotate the robot.
  It will keep rotating with the velocity. The velocity will decrease
  to zero in next "beeStop()".\item {\bf bool
    beeSetTransAcceleration(float acceleration)} - sets the
  translation acceleration of a BeeSoft robot.  The {\it
    "acceleration"} argument specifies the acceleration to set. The
  measure of acceleration is centimeter per square second. Positive
  value means positive acceleration, while negative value means
  negative acceleration.\item {\bf bool beeSetRotAcceleration(float
    acceleration)} - sets the rotation acceleration of a BeeSoft
  robot.  The {\it "acceleration"} argument specifies the acceleration
  to set. The measure of acceleration is degree per square second.
  Positive value means clockwise acceleration, while negative value
  means counterclockwise acceleration.\item {\bf robot\_position *
    beeGetCurrentPosition()} - returns current position of a BeeSoft
  robot. The position is defined as the data structure: {\it typedef
    struct \{    float pos\_x ; float pos\_y ; float orientation ; \}
    robot\_position ;}beeGetCurrentPosition() returns a pointer to the
  {\it robot\_position}, or NULL if an error occurs.\item {\bf
    robot\_speed * beeGetCurrentSpeed()} - returns current speed of a
  BeeSoft robot. The speed is defined as the data structure:{\it
    typedef struct \{ float trans\_speed ; float rot\_speed ; \}
    robot\_speed ;}beeGetCurrentSpeed() returns a pointer to the {\it
    robot\_speed}, or NULL if an error occurs.\item {\bf bool
    beeApproachRelative(float x, float y)} - moves a
  BeeSoft robot to a relative position. The first argument is the
  distance you want the robot to move perpendicular to its current
  heading. The second argument is the distance you want the robot to
  move along its current heading.  The measure is centimeter. 
    The robot will move toward the expected point that you specify.
  The robot can do the collision avoidance and try to reach the
  destination. However, because this function does not communication
  with Planner server, it can not plan its optimal route. You could
  use it to do local movements. \item {\bf bool
    beeApproachAbsolute(float x, float y) }- move a BeeSoft robot to
  an absolute position. It is the position in the map coordinate
  system. The {\it x} and {\it y} arguments specify the coordinates.
  The measure is centimeter. The robot will move to the expected
  point. The functions relies on LOCALIZE module to do the
  transformation between the robot's base coordinate system and the
  map coordinate system. You must run LOCALIZE first. \item {\bf bool
    beeGotoRelative (float x, float y)} - moves a BeeSoft robot to a
  relative position. The arguments {\it x} and {\it y} specify the
  destination of relative coordinate. The measure is centimeter. The
  robot will move to the expected point. With Planner running, the
  robot can find out the optimal path to move. \item {\bf bool
    beeGotoAbsolute(float x, float y)} - moves a BeeSoft robot to an
  absolute position. beeGotoAbsolute() uses Planner to find out the
  optimal path to the destination, while beeApproachAbsoulte() uses
  only colliServer's collision avoidance function.
\end{enumerate}{\bf \par Sensor functions: }functions that retrieve
the sensor readings\par \begin{enumerate}{\bf \item bool
    beeSonarStartRegularUpdate()} - triggers to regularly update sonar
  readings of a BeeSoft robot. User process can register a call-back
  function by beeSonarRegisterCallback() before the function. The
  callback function will then be invoked regularly and get an array of
  sonar readings as a parameter. Otherwise, sonar readings are
  processed by BeeSoft default callback function without notifying
  user programs. To get sonar data by invoking beeGetSonarValue(),
  beeSonarStartRegularUpdate() must be executed first.\item {\bf bool
    beeSonarStopRegularUpdate()} - stops the regular update operation.
\item {\bf float * beeGetSonarValue(int * number\_of\_reading)} -
  returns collected sonar readings of a BeeSoft robot. {\it
    "number\_of\_reading"} argument specifies a reference to an
  integer variable. The function returns readings only after the
  program runs beeSonarStartRegularUpdate(). It returns NULL if an
  error occurs. After the function, {\it
    \&quot;number\_of\_reading\&quot;} will get the number of readings
  in the returned array.\item {\bf bool beeIRStartRegularUpdate()} -
  triggers to regularly update infrared readings of a BeeSoft robot.
  User process can register a call-back function by
  beeIRRegisterCallback() before the function. The callback function
  will then be invoked regularly and get an array of infrared readings
  as a parameter. Otherwise, infrared data are processed by BeeSoft
  default callback function without notifying user program. To get
  infrared data by invoking beeGetIRValue(), beeIRStartRegularUpdate()
  must be executed first.\item {\bf bool beeIRStopRegularUpdate()} -
  stops the regular update operations. \item {\bf int *
    beeGetIRValue(int rowno , int * number\_or\_reading )} - returns
  collected infrared readings of a BeeSoft robot. In the {\it
    \&quot;number\_of\_reading"} argument the programmer give a
  reference to an integer variable. This function returns readings
  only after the program runs beeIRStartRegularUpdate().You can
  specify a row of infrared sensors in the {\it "rowno"} argument. In
  "bee.h", "UPPERROW", \&quot;LOWERROW", "DROW" are already defined as
  three rows. The function returns NULL if an error occurs. After the
  function, {\it \&quot;number\_of\_reading\&quot;} will get the
  number of readings in the returned array.\item {\bf bool
    beeLaserStartRegularUpdate()} - triggers to regularly update laser
  data of a BeeSoft robot. User process can register a call-back
  function by beeLaserRegisterCallback() before the function. The
  callback function will then be invoked regularly and get an array of
  laser readings as a parameter. Otherwise, laser data are processed
  by system defined callback function without notifying user program.
  To get laser data by invoking beeGetLaserValue(),
  beeLaserStartRegularUpdate() must be executed first.\item {\bf bool
    beeLaserStopRegularUpdate() - }stops the regular update operations
\item {\bf int * beeGetLaserValue(int whichone , int *
    number\_or\_reading )} - returns collected laser readings of a
  BeeSoft robot. In the {\it \&quot;number\_of\_reading"} argument the
  programmer give a reference to an integer variable. This function
  returns valid readings only after the program runs
  beeLaserStartRegularUpdate(). You can specify a row of laser sensors
  in the {\it "whichone"} argument. In "bee.h", "FRONT\_LASER",
  "BACK\_LASER" are already defined as front and back laser. The
  function returns NULL if an error occurs. After the function, {\it
    \&quot;number\_of\_reading\&quot;} will get the number of readings
  in the returned array. \end{enumerate}{\bf \par Camera Library:
  }functions that retrieve camera images\par \begin{enumerate}{\bf
  \item bool beeCameraSaveFile(int numGrabber, char * filename , int
    num )}- requests to cameraServer to save images in a file. {\it
    "numGrabber"} specifies the grabber number.  {\it "filename"
    }specifies the name of file to which the images will save. And
  {\it "num"} specifies the number of images to request. -1 means to
  save continuous images and 0 means to stop.\item {\bf bool
    beeCameraLoadFile(int numGrabber , char * filename ) - }requests
  to cameraServer to load images from a file. {\it "numGrabber"}
  specifies the grabber number. {\it "filename" }specifies the name of
  file from which the images will be retrieved. \item {\bf int
    beeCameraRequestShmId(int numGrabber ) - }requests for an image
  from cameraServer. {\it "numGrabber"} specifies the grabber number.
  This function uses shared memory to load images. It requests to
  cameraServer and blocks. cameraServer will load an{\bf }image in a
  shared memory space and reply with a shared memory Id. The function
  returns the shared memory id, or -1 if an error occurs. \item {\bf
    CameraImageType* beeCameraRequestImage(int numGrabber , int
    orig\_image\_xmin , int orig\_image\_ymin , int orig\_image\_xsize
    , int orig\_image\_ysize , int return\_image\_xsize , int
    return\_image\_ysize ) - }requests for images captured from
  cameraServer. The meanings of the arguments are: {\it "numGrabber"}
  specifies a grabber number.{\it "orig\_image\_xmin"} and {\it
    "orig\_image\_ymin"} give an location of a block inside the image.
  They specify the x, y coordinate for the (0,0) of the block. That
  means you could request for only a fraction of a camera image
  instead of a whole one.{\it \&quot;orig\_image\_xsize\&quot;} and
  {\it "orig\_image\_ysize" }give the size of a block you want. {\it
    "return\_image\_xsize"} and {\it "return\_image\_ysize"} give the
  size of a return image, which could reduce the amount of data needed
  to be transferred.  The function returns a pointer to
  "CameraImageType" data structure.  typedef struct \{ int xsize,
  ysize; unsigned char *red; unsigned char *green; unsigned char
  *blue; int numGrabber; \} CameraImageType; The function will return
  a pointer to CameraImageType, or NULL if an error occurs.\item {\bf
    bool * beeCameraSubscribe(int numGrabber , int numberOfImage , int
    image\_xmini , int image\_ymin , int image\_xsize , int
    image\_ysize , int return\_image\_xsize , int
    return\_image\_ysize)} - subscribes to images captured from the
  cameraServer. The meanings of the arguments are: {\it "numGrabber"}
  specifies a grabber number you want to subscribe to. {\it
    "numberOfImage}" specifies the request for every n-th frames. {\it
    "image\_xmin"} and {\it "image\_ymin"} give an location of a block
  inside the image. They specify (0,0) of the block.{\it
    "image\_xsize"} and {\it "image\_ysize"} give the size of a block
  inside the image. {\it "return\_image\_xsize"} and {\it
    "return\_image\_ysize"} give the size of a return image, which
  could reduce the amount of data needed to be transferred.  After the
  function is executed, a callback handler will be invoked when every
  n-th frame is captured. Programmers can register their own callback
  function by invoking beeCameraImageRegisterCallback().\item {\bf int
    * beeCameraUnsubscribe(int numGrabber ) - }unsubscribes frames
  from a cameraServer. \end{enumerate}{\bf \par Callback
  functions: }functions register callback functions\par
\begin{enumerate}{\bf \item bool beeSonarRegisterCallback(void (*
    callbackHandler )(float*) = NULL)} - registers a callback function
  to collect sonar readings of a BeeSoft robot.  {\it
    "callbackHandler"} argument specifies the location of the callback
  function. Then the program can run beeSonarStartRegularUpdate() to
  activate the update operations. The callback handler should have the
  signature:{\it void handler\_name(float * sonar );}Note that the
  sonar data is an array of 24 float values. Programmers can use{\it
    sonar[0-23]} inside the scope of callback handler to retrieve the
  values. You do not need to and should not release the memory space
  referenced by {\it sonar. }You can invoke beeSonarRegisterCallback()
  without any argument to unregister your own callback handler.\item
  {\bf bool beeIRRegisterCallback(void (* callbackHandler )(int*,
    int*, int*) = NULL ) - } registers a callback function to collect
  infrared readings of a BeeSoft robot. {\it "callbackHandler"}
  argument specifies a reference to the callback function. Then the
  program can run beeIRStartRegularUpdate() to activate the update
  operations. The callback handler should have the signature: {\it
    void handler\_name(int * irupperrow , int * irlowerrow , int *
    irdrow );} Note that the infrared data are three arrays of integer
  readings. Programmers can use {\it irupperrow[0-23],
    irlowerrow[0-23], irdrow[0-7]} inside the scope of callback
  handler to retrieve the values. You do not need to and should not
  release the memory space referenced by the tree parameters. You can
  invoke beeIRRegisterCallback() without any argument to unregister
  your own callback handler.\item {\bf bool
    beeLaserRegisterCallback(void (* callbackHandler )(int, float,
    float, int*, int, float, float, int*) = NULL ) - }registers a
  callback function to collect laser readings of a BeeSoft robot. {\it
    "callbackHandler"} argument specifies a reference to the callback
  function. After that, the program can run
  beeLaserStartRegularUpdate() to activate the update operations. The
  callback handler should have the signature:{\it void
    handler\_name(int laser\_f\_numberOfReadings , float
    laser\_f\_angleResolution , float laser\_f\_startAngle , int *
    laser\_f\_reading , int laser\_r\_numberOfReadings , float
    laser\_r\_angleResolution , float laser\_r\_startAngle , int *
    laser\_r\_reading );}Note that the laser data are arrays of int
  values. Programmers can use {\it laser\_f\_numberOfReadings} to get
  the number of readings of front laser and {\it
    laser\_r\_numberOfReadings} to get the number of readings of rear
  laser; and then use {\it laser\_f\_reading[index]} and {\it
    laser\_r\_reading[index]} inside the scope of callback handler to
  retrieve the values. You do not need to and should not release the
  memory space referenced by the laser\_f\_reading parameters. You can
  invoke beeLaserRegisterCallback() without any argument to unregister
  your own callback handler.\item {\bf bool
    beeCameraRegisterCallback(void (* callbackHandler
    )(cameraImageType *) = NULL) - }registers a callback function to
  collect camera images from a BeeSoft camera server.{\it
    \&quot;callbackHandler"} argument specifies a reference to the
  callback function. After that, the program can run
  beeCameraSubscribe() to activate the update operations.  The
  callback handler should have the signature:{\it void
    handler\_name(cameraImageType * image ); cameraImageType} is
  defined in the description of beeCameraRequestImage(). You can
  invoke beeCameraRegisterCallback() without any argument to
  unregister your own callback handler.
\end{enumerate}\end{document}

